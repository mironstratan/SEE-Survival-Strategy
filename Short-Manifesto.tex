\documentclass{article}
\usepackage[utf8]{inputenc}
\usepackage{apacite}
\usepackage{dirtytalk}
\topmargin 0.0cm
\oddsidemargin 0.0cm
\textwidth 16cm 
\textheight 21cm
\footskip 1.0cm

\usepackage{hyperref}
\hypersetup{
    colorlinks=true,
    linkcolor=blue,
    filecolor=magenta,      
    urlcolor=cyan,
}
 
\urlstyle{same}

\title{Project Manifesto}
\author{Miron Stratan}

\begin{document}

\maketitle


\section*{The \say{Dissimulation as a survival strategy since the Neolithic in the wider Southeast Europe} project is looking for subject matter experts interested in explaining how genes and ideas tend to cling for millennia to the mountains in Central and Southeastern Europe.}

Why give attention to this project? Maybe you are interested in this region, or your work is connected to it, or the arguments in your field need a different interpretation. In the end, if proven right, the proposed theory could give a new viewpoint on the thinking of past and present people inhabiting this part of Europe.

\section*{The gist}

The mountains, valleys and plains enclosed by the Carpathians, the Dinaric Alps, the Rhodopes, the Pindos, and the Balkans create a multitude of refuge options for people to thrive in even during climatic stress and under migratory pressure. As a consequence, this region allows for genetic continuity and the persistence of specific subsistence strategies.

While moving through this region, migration waves have to slow down in order to negotiate mountain passes and to deal with the slightly different climate and environment on the other side. Mediterranean farmers had to stop for generations to adapt their crops and domesticated animals to temperate climate once they reached the Danube, while steppe riders had to dismount in order to cross the forests toward mountain pastures or to reach the Carpathian basin steppe. In multiple occasions, that slow down meant time enough to mingle with the locals and to exchange cultural elements and genes.

Since this region is smacked right at the crossroads between Central Europe and Central and Southwestern Asia, this story  repeated for millennia with each movement of people and ideas, thus consolidating, locally, a specific form of self-preservation. The initiator of the project contends that the current cultures within the region potentially maintain dissimulation as a part of that protection strategy, along with other prehistoric cultural relics.


\section*{Get in touch}
Please send proposals, comments and criticism to Miron Stratan at~miron.stratan@endava.com or via the project page:


\noindent
\footnotesize
\url{https://www.researchgate.net/project/Dissimulation-as-a-survival-strategy-in-South-East-Europe-since-Neolithic} 


\end{document}
